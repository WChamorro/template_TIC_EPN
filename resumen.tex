\chapter*{RESUMEN}
\addcontentsline{toc}{chapter}{RESUMEN} % Adds RESUMEN to the Table of Contents
\justifying
(Máximo 250 palabras)
En el presente trabajo de integración curricular se plantea como fin el control de un brazo robótico de 5 grados de libertad para tareas de paletización. El brazo que se utilizará fue ensamblado en el TIC “Ensamblaje y Control de un Robot con Arquitectura Antropomórfica de cinco grados de libertad”, en este TIC se realizó el control tanto en alto nivel como en bajo nivel. Tomando como base dichos controladores se realizó la adaptación necesaria para que el robot realice tareas de paletización. Además, se desarrolló una interfaz gráfica en MATLAB-Simulink donde se encuentra implementada la cinemática para el control del movimiento angular de cada uno de los motores, en dicha interfaz gráfica el usuario debe ingresar las coordenadas de inicio y fin para realizar las tareas de paletización. Adicionalmente para complementar al robot ensamblado se manufacturó un efector final tipo pinza de 2 dedos cuyos planos de diseño se obtuvieron del proyecto de investigación PIMI-14-04. Al culminar con este proyecto, el laboratorio de Sistemas de control del DACI contará con un prototipo de un brazo robótico de 5 GDL que tenga la capacidad de realizar tareas de paletización para realizar prácticas de laboratorio de la asignatura de Robótica. Estos resultados consolidados representarán no solo la culminación exitosa de los esfuerzos invertidos en el proyecto, sino también una contribución significativa para fortalecer las capacidades del laboratorio y facilitar la realización de prácticas educativas aplicadas en el entorno de la robótica.

\vspace{1cm}
\noindent \textbf{PALABRAS CLAVE:} antropomórfico, paletización, grados de libertad.

\newpage
