 
\chapter*{\centering ABSTRACT}
\addcontentsline{toc}{chapter}{ABSTRACT} % Adds RESUMEN to the Table of Contents


\justifying
(Máximo 250 palabras)
This curricular integration project aims to control a 5-degree-of-freedom robotic arm for palletizing tasks. The arm to be used was assembled in the TIC project "Assembly and Control of a Robot with an Anthropomorphic Architecture of Five Degrees of Freedom," in which both high-level and low-level control were implemented. Based on these controllers, the necessary adaptation was made for the robot to perform palletizing tasks. Additionally, a graphical interface in MATLAB-Simulink was developed, where the kinematics for controlling the angular movement of each motor is implemented. In this graphical interface, the user must enter the start and end coordinates to carry out the palletizing tasks. Furthermore, to complement the assembled robot, a two-finger gripper-type end effector was manufactured, with design plans obtained from the research project PIMI-14-04. Upon completion of this project, the Control Systems Laboratory of DACI will have a prototype of a 5-DOF robotic arm capable of performing palletizing tasks for laboratory practices in the Robotics course. These consolidated results will not only represent the successful culmination of the efforts invested in the project but also a significant contribution to strengthening the laboratory's capabilities and facilitating the execution of applied educational practices in the field of robotics.

\vspace{1cm}
\noindent \textbf{KEYWORDS:} anthropomorphic, palletizing, degrees of freedom.

\newpage
